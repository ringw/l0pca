\documentclass{article}
\usepackage{amsmath}
\usepackage{amssymb}

\title{Triangularize Eigenvalues}

\begin{document}
	
\maketitle

We have $\Sigma$, which is a diagonal singular values matrix. Our original, full-rank data matrix is augmented with a column of zeros, so $\Sigma$ has rank $n-1$ because the last row and column are zero. For a new column of data, there is a $\mathbb{R}^n$ vector update to the last column of $\Sigma$, and new additional left and right singular vector, to reproduce the data.

We can readily compute the sparse matrix $\Sigma^{\prime T} \Sigma^\prime = D^\prime$:

$$
D^\prime =
\left(\begin{matrix}
d_1 & 0 & 0 & w_1 \\
0 & d_2 & 0 & w_2 \\
0 & 0 & d_3 & w_3 \\
w_1 & w_2 & w_3 & w_4
\end{matrix}\right)
$$

We can use Givens rotations on both sides to efficiently rotate the entries of $D^\prime$ to upper triangular. We want $n-1$ matrices on the LHS which progressively push down the $w$ entries in the last column. The last column of $D^\prime$ may be isolated, and the effect of the series of Givens rotations will be computed in parallel.

Now, suppose that we have applied a Givens rotation to zero out $D^\prime_{in}$. We need to consider two progressive 2x2 rotations affecting certain diagonal entries: $d_i \to d_i^\prime \to d_i^{\prime\prime}$

First, $d_1$ will only have one 2x2 rotation touching it, so initialize $d_1^\prime = d_1$.

Next, consider $d_i, i > 1$. We have an angle: $\sin \theta_i = \frac{w_{i-1}^\prime}{w_i^\prime}$. For $w_2^\prime$, we formed the hypotenuse of the two entries, and by reversing the angle (which is in the first quadrant; both positive), we push $\sin \frac{w_1}{\sqrt{w_1^2+w_2^2}}$ to zero.

Apply the 2x2 rotation to the 2x2 block:

$$
\left(
\begin{matrix}
\cos \theta_i & -\sin\theta_i \\
\sin\theta_i & \cos\theta_i
\end{matrix}
\right)
\left(\begin{matrix}
d_{i-1} & 0 \\
0 & d_i \end{matrix}\right)
=
\left(\begin{matrix}
	\ell_1 & r_1 \\ \ell_2 & r_2
\end{matrix}\right)
=
\left(\begin{matrix}
	d_{i-1}^\prime\cos\theta_i
	&
	-d_i \sin\theta_i
	\\
	d_{i-1}^\prime\sin\theta_i
	&
	d_i\cos\theta_i
\end{matrix}\right)
$$

Before writing our updates ($d_{i-1}^{\prime\prime}$ and $d_i^\prime$), we need $r_1$ to go to zero. Apply a 2x2 rotation matrix to the adjoint:

$$
\left(\begin{matrix}
	\cos \phi_i & -\sin\phi_i \\
	\sin\phi_i & \cos\phi_i
\end{matrix}
\right)
\left(\begin{matrix}
	\ell_1 & r_1 \\ \ell_2 & r_2
\end{matrix}\right)^T
=
\left(\begin{matrix}
	d_{i-1}^{\prime\prime}
	&
	0
	\\
	* & d_i^\prime
\end{matrix}\right)^T
$$

We expect: $d_{i-1}^{\prime\prime} = \sqrt{\ell_1^2 + r_1^2}$

Apply the reverse rotation matrix to the transposed entries:

$$
\left(\begin{matrix}
	\cos \phi_i & -\sin(-\phi_i) \\
	\sin(-\phi_i) & \cos\phi_i
\end{matrix}
\right)
\left(\begin{matrix} d_{i-1}^{\prime\prime
	}
\\ 0 \end{matrix}\right)
=
\left(\begin{matrix}
	\ell_1 \\ r_1 \end{matrix}\right)
$$

Therefore: $\cos\phi_i = \frac{\ell_1}{d_{i-1}^{\prime\prime}}$

Our last step is to solve for $d_i^\prime$:

$$
\left(\begin{matrix}
	\cos \phi_i & -\sin\phi_i \\
	\sin\phi_i & \cos\phi_i
\end{matrix}
\right)
\left( \begin{matrix} \ell_2 \\ r_2 \end{matrix}\right)
=
\left( \begin{matrix} * \\ d_i^\prime \end{matrix} \right)
$$

Using the results of the two rotations:

$$
d_i^\prime =
\ell_2 \sin\phi_i + r_2 \cos\phi_i =
d_{i-1}^\prime \sin\theta_i \sin\phi_i + d_i \cos\theta_i \cos\phi_i
$$

Note: $\sin \phi_i = r_1 / \sqrt{\ell_1^2 + r_1^2}
= -d_i w_{i-1} / \sqrt{w_i^{\prime 2} (d_{i-1}^{\prime2} \cos^2 \theta_i + d_i^2 \sin^2 \theta_i)}
$

We need an approximation before continuing. Say that we will place the diagonal in ascending order of magnitude, and the value $w_{i-1}^\prime$ which we want to push towards zero keeps growing relative to the next $w_i$ ($|\sin\theta_i|$ is growing, $|\sin\theta_i| > |\cos\theta_i|$). Therefore, it makes the most sense to disregard $d_{i-1}^{\prime2} \cos^2\theta_i$.

$$
\sin \phi_i
\approx
-\frac{d_iw_{i-1}}{
	w_i^\prime d_i \sin\theta_i
}
=
-\frac{w_{i-1}}{w_{i-1}^\prime}
$$

$$
\cos \phi_i
\approx
\frac{d_{i-1}^\prime \cos\theta_i}{w_i^\prime d_i \sin\theta_i}
=
\frac{d_{i-1}^\prime w_i}{d_i w_i^\prime w_{i-1}^\prime}
$$

Plug this into the closed-form for $d_i^\prime$:

$$
d_i^\prime
=
-\frac{d_{i-1}^\prime w_{i-1}}{d_i w_{i-1}^\prime} \frac{w_{i-1}^\prime}{w_i^\prime}
+
d_i \frac{w_i}{w_i^\prime} \frac{d_{i-1}^\prime w_i}{d_i w_i^\prime w_{i-1}^\prime}
=
-\frac{d_{i-1}^\prime w_{i-1}}{d_i w_i^\prime} + \frac{d_{i-1}^\prime w_i^2}{w_i^{\prime 2} w_{i-1}^\prime}
$$

We have each term increasing by a ratio:

$$
d_i^\prime = \left( \frac{w_i^2}{w_i^{\prime2} w_{i-1}^\prime} - \frac{w_{i-1}}{d_i w_i^\prime} \right) d_{i-1}^\prime
$$

% TODO: We have strict constraints, e.g. di < di' < d(i+1). The purpose of this solution is to produce an initial guess strictly contained in each window for each eigenvalue. So, probably want to not actually approximate it.

\end{document}