\documentclass{article}
\usepackage{amsmath}
\usepackage{amssymb}

\title{Triangularize Eigenvalues}

\begin{document}
	
\maketitle

We have $A^2$ which was a diagonal matrix of size $n-1$ ($A^2_i, i < n$), and is augmented by $v^2_i, 1 \le i \le n$ as last row/column symmetric entries.

In one iteration, we update $v^2$, but in the next iteration, this entry will go to zero.

$$v^{2\prime}_2 = \pm \sqrt{\left(v^{2\prime}_1\right)^2 + \left(v^2_2\right)^2}$$

Calculate $\left(v_i^{2\prime}\right)^2$ using cumsum of $\left(v_i^2\right)^2$ (originally this matrix was formed by squaring another matrix with augmented vector $v$, now we are squaring again).

We wanted $\sin \theta_i$ to drive $v_i^2$ to zero. Then: $\cos \theta_i = \frac{v_2^{2\prime}}{v_2^2}$. $\theta_i$ should have appropriate sign (modulus of $\pi$) so that $\left|v_2^{2\prime}\right|$ grows instead of shrinking.

Consider $A^2_i$ when we update $v^2_i$. There is an intermediate step where $A^2_i$ goes to $A^2_i \cos\theta_i$, and there is an upper off-diagonal in the row above: $A^2_i \sin\theta_i$ (upper off-diagonal must be avoided). We have a 2x2 block:

$$
M
\left(\begin{matrix}
	A^{2^\prime}_{i-1} & 0
	\\
	0 & A^{2}_{i}
\end{matrix}\right)
=
\left(\begin{matrix} A^{2^\prime}_{i-1} \sin\theta_i & A^2_i \sin\theta_i \\ A^{2^\prime}_{i-1} \cos\theta_i & A^2_i \cos\theta_i \end{matrix}\right)$$

Next, we need the top-right (upper off-diagonal) to go to zero. The angles would be precomputed in parallel after the first computation, by lining up $A_{i-1}^{2^\prime}$ and $A_i^2$. Applying a Givens rotation on the right (rotation of two elements in the row vector, broadcasted against each row of the matrix), the result is:

$$\left(\begin{matrix}
	A^{2^\prime}_{i-1} \sin\theta_i & A^2_i \sin\theta_i \\ A^{2^\prime}_{i-1} \cos\theta_i & A^2_i \cos\theta_i
\end{matrix}\right)
\left(\begin{matrix}
	\cos\phi_i & -\sin\phi_i \\
	\sin\phi_i & \cos\phi_i
\end{matrix}\right)
$$

The first row is discarded

\subsection*{try again}

We have $\Sigma$, which is a diagonal singular values matrix. Our original, full-rank data matrix is augmented with a column of zeros, so $\Sigma$ has rank $n-1$ because the last row and column are zero. For a new column of data, there is a $\mathbb{R}^n$ vector update to the last column of $\Sigma$, and new additional left and right singular vector, to reproduce the data.

We can readily compute the sparse matrix $\Sigma^{\prime T} \Sigma^\prime = D^\prime$:

$$
D^\prime =
\left(\begin{matrix}
d_1 & 0 & 0 & w_1 \\
0 & d_2 & 0 & w_2 \\
0 & 0 & d_3 & w_3 \\
w_1 & w_2 & w_3 & w_4
\end{matrix}\right)
$$

We can use Givens rotations on both sides to efficiently rotate the entries of $D^\prime$ to upper triangular. We want $n-1$ matrices on the LHS which progressively push down the $w$ entries in the last column. The last column of $D^\prime$ may be isolated, and the effect of the series of Givens rotations will be computed in parallel.

Now, suppose that we have applied a Givens rotation to zero out $D^\prime_{in}$. We need to consider two progressive 2x2 rotations affecting certain diagonal entries: $d_i \to d_i^\prime \to d_i^{\prime\prime}$

First, $d_1$ will only have one 2x2 rotation touching it, so initialize $d_1^\prime = d_1$.

Next, consider $d_i, i > 1$. We have an angle: $\sin \theta_i = \frac{w_1}{w_2^\prime}$. For $w_2^\prime$, we formed the hypotenuse of the two entries, and by reversing the angle (which is in the first quadrant; both positive), we push $\sin \frac{w_1}{\sqrt{w_1^2+w_2^2}}$ to zero.

Apply the 2x2 rotation to the 2x2 block:

$$
\left(
\begin{matrix}
\cos \theta_i & -\sin\theta_i \\
\sin\theta_i & \cos\theta_i
\end{matrix}
\right)
\left(\begin{matrix}
d_{i-1} & 0 \\
0 & d_i \end{matrix}\right)
=
\left(\begin{matrix}
	\ell_1 & r_1 \\ \ell_2 & r_2
\end{matrix}\right)
=
\left(\begin{matrix}
	d_{i-1}^\prime\cos\theta_i
	&
	-d_i \sin\theta_i
	\\
	d_{i-1}^\prime\sin\theta_i
	&
	d_i\cos\theta_i
\end{matrix}\right)
$$

Before writing our updates ($d_{i-1}^{\prime\prime}$ and $d_i^\prime$), we need $r_1$ to go to zero. Apply a 2x2 rotation matrix to the adjoint:

$$
\left(\begin{matrix}
	\cos \phi_i & -\sin\phi_i \\
	\sin\phi_i & \cos\phi_i
\end{matrix}
\right)
\left(\begin{matrix}
	\ell_1 & r_1 \\ \ell_2 & r_2
\end{matrix}\right)^T
=
\left(\begin{matrix}
	d_{i-1}^{\prime\prime}
	&
	0
	\\
	* & d_i^\prime
\end{matrix}\right)^T
$$

We expect: $d_{i-1}^{\prime\prime} = \sqrt{\ell_1^2 + r_1^2}$-

Apply the reverse rotation matrix to the transposed entries:

$$
\left(\begin{matrix}
	\cos \phi_i & -\sin(-\phi_i) \\
	\sin(-\phi_i) & \cos\phi_i
\end{matrix}
\right)
\left(\begin{matrix} d_{i-1}^{\prime\prime
	}
\\ 0 \end{matrix}\right)
=
\left(\begin{matrix}
	\ell_1 \\ r_1 \end{matrix}\right)
$$

Therefore: $\cos\phi_i = \frac{\ell_1}{d_{i-1}^{\prime\prime}}$

Our last step is to solve for $d_i^\prime$:

$$
\left(\begin{matrix}
	\cos \phi_i & -\sin\phi_i \\
	\sin\phi_i & \cos\phi_i
\end{matrix}
\right)
\left( \begin{matrix} \ell_2 \\ r_2 \end{matrix}\right)
=
\left( \begin{matrix} * \\ d_i^\prime \end{matrix} \right)
$$

Using the results of the two rotations:

$$
d_i^\prime =
d_{i-1}^\prime \sin\theta_i \sin\phi_i + d_i \cos\theta_i \cos\phi_i
$$

$$
d_i^\prime
=
d_{i-1}^\prime
\frac{w_{i-1}}{w_i^\prime}
\frac{(-d_i w_{i-1}/w_{i}^{\prime})}{\sqrt{
		d_{i-1}^{\prime2} \cos^2 \theta_i
		+ d_i^2 \sin^2 \theta_i
	}}
+
d_i
\frac{w_i}{w_i^\prime}
\frac{d_{i-1}^{\prime2} w_i/w_i^\prime}{\sqrt{
		d_{i-1}^{\prime2} \cos^2 \theta_i
		+ d_i^2 \sin^2 \theta_i
}}
$$

$$
d_i^\prime
=
\frac{
	d_{i-1}^{\prime2} d_i w_i^2 - d_{i-1}^\prime d_i w_{i-1}^2
}{}
$$

\end{document}